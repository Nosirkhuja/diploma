% !TEX TS-program = pdflatexmk
\documentclass[12pt,fleqn]{article}

\usepackage[utf8]{inputenc}
\usepackage[T2A]{fontenc}
\usepackage{amssymb,amsmath,mathrsfs,amsthm}
\usepackage[russian]{babel}
\usepackage{graphicx}
\usepackage[footnotesize]{caption2}
\usepackage{indentfirst}
%\usepackage[ruled,section]{algorithm}
%\usepackage[noend]{algorithmic}
%\usepackage[all]{xy}

% Ïàðàìåòðû ñòðàíèöû
\textheight=24cm
\textwidth=16cm
\oddsidemargin=5mm
\evensidemargin=-5mm
\marginparwidth=36pt
\topmargin=-2cm
\footnotesep=3ex
\setlength{\parskip}{6pt}
%\flushbottom
\raggedbottom
\tolerance 3000
% ïîäàâèòü ýôôåêò "âèñÿ÷èõ ñòpîê"
\clubpenalty=10000
\widowpenalty=10000
\renewcommand{\baselinestretch}{1.1}
\renewcommand{\baselinestretch}{1.5}
\linespread{1.5}
%äëÿ ïå÷àòè ñ áîëüøèì èíòåðâàëîì
\newcommand{\anonsection}[1]{\section*{#1}\addcontentsline{toc}{section}{#1}}

\begin{document}
	
	\begin{titlepage}
		\begin{center}
			Московский государственный университет имени М.В. Ломоносова\\
			Факультет вычислительной математики и кибернетики
			
			\bigskip
			{\includegraphics[scale=0.8]{msu}}
			\bigskip
			
			
			\textsf{\large\bfseries
				\\[10mm]
				ВЫПУСКНАЯ КВАЛИФИКАЦИОННАЯ РАБОТА
			}\\[1mm]
			
			\textsf{\large\bfseries
				\\[15mm]
				Реализация программного модуля суммирования сейсмической миграции с использованием графических ускорителей
			}\\[10mm]
				
			\begin{flushright}
				\parbox{0.55\textwidth}{
					
					
					
					Выполнил:\\
					\\
					\emph{Хомидов Носирхуджа Хомидович}\\[10mm]
					Научный руководитель:\\
					\\
				\emph{к.ф.-м.н.Левченко Вадим Дмитриевич}
				}
			\end{flushright}
			
			\vspace{\fill}
			{\large Москва, 2022}
			\end{center}
			
	
	\end{titlepage}
	
	\newpage
	\renewcommand{\contentsname}{Содержание}
	\tableofcontents
	\newpage
	\begin{abstract}
	\end{abstract}
	\section{Введение}
{\bfАктуальность темы}\\
Сейсморазведка - это метод геофизического исследования земной коры и его геологической среды, который основан на анализе естественных и искусственно возбужденных упругих волн и их поведении при встрече с горными породами. Основная задача
сейсморазведки -- сейсмическая миграция, которая  
состоит в реконструкции 3-мерного сейсмического изображения
глубинных неоднородностей изучаемой земной среды. Получаемое с
помощью миграции 3-мерное сейсмическое изображение используют
совместно со скважинными данными в процессе их комплексной
геологической интерпретации, результатом которой является цифровая
геологическая модель месторождения. В настоящее время наиболее востребованной является 3-мерная
глубинная миграция до суммирования, учитывающая преломление
сейсмических волн в фоновой модели неоднородной земной среды. Объем
исходных сейсмических данных, используемых для построения 3-мерного
сейсмического изображения, может достигать порядка 10 ТБ цифровой
информации, а наиболее экономичное приближенное решение задачи
методом Кирхгофа требует порядка $10^{17}$ арифметических операций. Это
ставит сейсмическую миграцию в ряд наиболее сложных и актуальных
научно-технических задач, для решение которой требуется использование
суперкомпьютерных вычислений. \\ 
{\bfОбзор источников} \\
В настоящее время известны два основных подхода к решению задачи сейсмической миграции. Первый основан на постановке и решении линеаризованной обратной задачи рассеяния волн, а второй – на обращенном волновом продолжении. Основное практическое значение продолжает иметь второй подход. Важный вклад в разработку метода обращенного волнового продолжения в начале 70-х годов прошедшего  века внесли американские исследователи Дж. Клаербоут и Дж. Газдаг, и
отечественные ученые Г.И. Петрашень и С.А. Нахамкин. Основанные на
обращенном волновом продолжении методы и алгоритмы сейсмической
миграции связаны с решением множества прямых 3-мерных задач на
распространения волн от точечных источников, число которых может
достигать порядка $10^5$. Наиболее экономичным является асимптотический
метод решения в предположении однолучевого распространения.
Последнее означает, что из источника в произвольную глубинную точку
среды приходит единственный луч по траектории, соответствующей
кратчайшему времени пробега. Основанные на этом приближении методы
сейсмической миграции сводятся к явным интегральным формулам и
называются миграцией методом Кирхгофа. В неоднородных сложнопостроенных средах предположение об однолучевом распространении
часто нарушается. Присутствие складок фронтов волн и областей их
фокусировки, называемых каустиками, требует корректного учета,
который недостижим в рамках однолучевой модели. Иной
асимптотический метод, известный под названием суммирования
гауссовых пучков, разработанный отечественным ученым М.М. Поповым,
до сих пор не вошел в число общеупотребительных средств сейсмической
миграции.\\
{\bfЦель работы}\\
Главной целью данной работы является получение корректного асимптотического решения с ограниченными амплитудами классической задачи волнового продолжения заданного граничного условия Дирихле..\\
{\bfЗадачи исследования}\\
Для достижения цели решаются следующие задачи:\\
- Моделирование волн от точечных источников в неоднородной среде.\\
- Разработка алгоритма и написание программного модуля миграционного преобразования методом обращённого волнового продолжения.\\
\section{Общий вид решения задачи сейсмической миграции до суммирования}
Следуя общей схеме рассуждений, представленной в [4], рассмотрим задачу для дивергентного вида уравнения Гельмгольца, описывающего распространение монохроматических колебаний в неоднородной среде:
	\begin{center}
		div($k$ grad $u$) $+ \omega^2\rho u=0$ \qquad  \qquad \qquad  \qquad  \qquad  \qquad (1)
	\end{center} 
	Решение задачи волнового продолжения в неоднородное полупространство:\
	\begin{center}
	$u(r,\omega)=-\frac{1}{4\pi} \iint \limits_{S} u_0(r_0, \omega) k(r_0) \frac{\partial G(r_0 | r; \omega)}{\partial \textbf{n}} dr_0$ \qquad  \qquad  \quad(2)
	\end{center} 
	где $u|_s=u_0$ - функция, задающая граничное условие Дирихле,  S - граница неоднородного полупространства $z>0$, $\omega$ – циклическая временная частота; $G(r_0 | r; \omega)$ - функция Грина;  $\frac{\partial}{\partial n}$ - производная по направлению внешней нормали к S; $k=k(r)$ и $\rho = \rho(r)$ заданные параметры неоднородной среды, определяющие локальную скорость распространения колебаний $\nu(r)= \sqrt{k(r)/\rho(r)}$; $dr_0$  – элемент поверхности интегрирования S. \\
	Введем обращенное время  $\tilde t=T-t$ запишем в системе координат обращенного времени функцию $\tilde u= \tilde u (r_s, r_g, \tilde t)$. Где $u(r_s, r_g, t) $ описывает результаты регистрации волн произвольно расположенными в полупространстве z > 0 приемниками в точках $r_g=(x_g, y_g,z_g)$ от источника в произвольных точках $r_s=(x_s, y_s,z_s)$. Зафиксируем некоторый произвольный источник s. Используя граничное условие $u_0=u_0(s,g, \omega) = u(r_s, r_g, t)|_{z_s=z_g=0}$ можно применить интегральную формулу волнового продолжения (2) и получить результат обращенного волнового продолжения в нижнее полупространство $z_g > 0$: \\
	$u(s,r_g;\omega)=-\frac{1}{4\pi} \iint \limits_{S_g} u_0(s, g; \omega) k(g) \frac{\partial G(g| r_g ; \omega)}{\partial \textbf{n}_s} dg$, \qquad  \qquad  \quad  \qquad  \qquad  \quad \qquad  \qquad \qquad  \qquad(3)
	
	 где $dg$ - элемент площади интегрирования, $n_g$ - внешняя нормаль к поверхности $S_g$. Полученное интегральное выражение представляет собой результат восстановления в произвольной точке $r_g$ полупространства $z_g > 0$ волнового поля по его измеренным значениям на поверхности наблюдений $S_g$ для произвольного фиксированного источника s. Зафиксируем точку $r_g$ и рассмотрим продолженное волновое поле  $ \tilde u (r_s, r_g, \tilde t)$ как функцию координат произвольного источника s.\\
	 {\bf Асимптотическое решение задач волнового продолжения и сейсмической миграции до суммирования}\\
	% \section{Постановка задачи}
	 Напишем  формальное асимптотическое приближение функции Грина из (2) в виде:
	 \begin{center}
	 $G(r_0 | r_*; \omega) \approx \sum \limits_{m}|A_m(r_0 | r_*)|e^{i\omega\tau_m(r_0|r_*)} e^{i Ind_m \text{sgn} \omega\pi/2}$, 
	 \end{center}
	  где $|A_m(r_0 | r_*)|e^{i\omega\tau_m(r_0|r_*)} e^{i Ind_m \text{sgn} \omega\pi/2}$ -одно из слагаемых лучевой функции Грина, отвечающих непрерывному участку фронта волны – «лепестку» с номером m, связанному с пучком лучей, исходящих из глубинной точки $r_*$, с фиксированной величиной накопленного индекса $Ind_m(r_0|r_*)$. $Ind_m(r_0|r_*)$ отвечает полному числу обращений в ноль якобиана $J\left(\mathbf{r}(\tau, \varphi, \theta) \mid \mathbf{r}_{*}\right)=\left(\frac{\partial \mathbf{r}}{\partial \tau}, \frac{\partial \mathbf{r}}{\partial \varphi}, \frac{\partial \mathbf{r}}{\partial \theta}\right)$ вдоль соответствующей лучевой трубки. 	$A_m(r_0| r_*)$ представляет собой лучевую амплитуду m-лепестка в точке $r_0$ выхода соответствующей лучевой трубки на поверхность наблюдений, имеющую вид: 
	  \begin{center}
	  $A\left(\mathbf{r} \mid \mathbf{r}_{*}\right)=\frac{1}{k_{*}}\left(\frac{\nu_{*} \rho_{*}|\sin \theta|}{\nu \rho\left[\frac{J\left(\mathbf{r} \mid \mathbf{r}_{*}\right)}{\nu}\right]}\right)^{1 / 2} .$ \qquad \qquad \qquad \qquad (4)
	\end{center}
	Отсюда получим:\\
	$\begin{gathered}
\frac{\partial G\left(\mathbf{r}_{0} \mid \mathbf{r}_{*} ; \omega\right)}{\partial \mathbf{n}} \approx\left(\sum_{m} i \omega\left|A_{m}\left(\mathbf{r}_{0} \mid \mathbf{r}_{*}\right)\right| e^{i \omega \tau\left(\mathbf{r}_{0} \mid \mathbf{r}_{*}\right)} e^{i     \operatorname{lnd}_{m} \operatorname{sgn} \omega \pi / 2} \nabla_{\mathbf{r}_{0}} \tau_{m}\left(\mathbf{r}_{0} \mid \mathbf{r}_{*}\right), \mathbf{n}\right)= \\
=i \omega \sum_{m}\left|A_{m}\left(\mathbf{r}_{0} \mid \mathbf{r}_{*}\right)\right| \frac{\cos \alpha_{m}}{v\left(\mathbf{r}_{0}\right)} e^{i \omega \tau_{m}\left(\mathbf{r}_{0} \mid \mathbf{r}_{*}\right)} e^{i \operatorname{Ind} \operatorname{sgn} \omega \pi / 2}
\end{gathered},\qquad (5)$\\
 $\alpha_m$ - угол между направлением выхода луча, принадлежащего m-лепестку, в точке $r_0$ поверхности S и нормалью n. Выполнив подстановку (5) в (2), получим:\\
$\begin{gathered}
u\left(\mathbf{r}_{*} ; \omega\right) \approx-\frac{1}{4 \pi} \iint \limits_{S} u_{0} k_{0} \sum_{m} i \omega\left|A_{m}\left(\mathbf{r}_{0} \mid \mathbf{r}_{*}\right)\right| \frac{\cos \alpha_{m}}{\nu\left(\mathbf{r}_{0}\right)} e^{i \omega \tau_{m}\left(\mathbf{r}_{0} / \mathbf{r}_{\mathbf{r}}\right)} e^{i \operatorname{lnd}_{m} \operatorname{sgn} \omega \pi / 2} d \mathbf{r}_{0} \approx \\
\approx-\frac{1}{4 \pi} \sum_{m} \iint \limits_{S_{m}} u_{0} k_{0} i \omega\left|A_{m}\left(\mathbf{r}_{0} \mid \mathbf{r}_{*}\right)\right| \frac{\cos \alpha_{m}}{\nu\left(\mathbf{r}_{0}\right)} e^{i \omega \tau_{m}\left(\mathbf{r}_{0} \mid \mathbf{r}_{\mathbf{r}}\right)} e^{i \operatorname{lnd}_{m} \operatorname{sgn} \omega \pi / 2} d \mathbf{r}_{0} \qquad \qquad \qquad \qquad(6)
\end{gathered}$
где $d \mathbf{r}_{0}$ - элемент площади поверхности $S ; S_{m}$ - участок поверхности, связанный с $m$-лепестком многозначного эйконала $\tau_{m}\left(\mathbf{r}_{0} \mid \mathbf{r}_{*}\right)$. Заметим, что входящая в (6) амплитуда $\left|A_{m}\left(\mathbf{r}_{0} \mid \mathbf{r}_{*}\right)\right|$ в соответствии с (4) является неограниченной функцией в окрестности каустик.\\
Выполнив обратное преобразование Фурье по времени (6), получим решение задачи волнового продолжения во временной области:
\begin{center}
	$u\left(\mathbf{r}_{*}, t\right) \approx-\frac{1}{4 \pi} \sum \limits_{m} \iint \limits_{S_m} k_{0}\left|\tilde{A}_{m}\left(\mathbf{r}_{0} \mid \mathbf{r}_{*}\right)\right| \frac{\cos \alpha_{m}}{\nu\left(\mathbf{r}_{0}\right)} H^{\text {Ind}_m}\left(u_{0}\right)_{t}^{\prime}\left(x_{0}, y_{0} ; t+\tau_{m}\left(\mathbf{r}_{0} \mid \mathbf{r}_{*}\right)\right) d x_{0} d y_{0}   \qquad (7)$
	\end{center} 
Выпишем выражение асимптотического решения задачи 3D-миграции до суммирования данных многократных перекрытий: 
{\footnotesize 
$$f\left(\mathbf{r}_{*}\right) \approx\left(\frac{1}{4 \pi}\right)^{2} \sum_{l, l'} \int \limits_{S_{l}} \int \limits_{S_{l'}} k_{\mathrm{s}}k_{\mathrm{g}} \frac{\cos \left(\alpha_{s}\right)_{l} \cos \left(\alpha_{g}\right)_{l'}}{\nu_{s} \nu_{g}}\left[A_{s}\right]_{l}\left[A_{g}\right]_{l'} H^{\operatorname{Ind}_l+\operatorname{Ind}_{l'}}\left(u_{0}\right)_{t}^{\prime}\left(\mathbf{s}, \mathbf{g} ;\left(\tau_{s}\right)_{l}+\left(\tau_{g}\right)_{l'}\right) d \mathbf{s} d \mathbf{g} $$ (8)} \\
где $S_{l}$ и $S_{l'}-$ участки площади поверхности, связанные с лепестками $l, l'$ и отвечающими им участками фронта волны с постоянным индексом для точечного источника, расположенного в $\mathbf{r}_{*}$. Суммирование выполняется по всем возможным сочетаниям лепестков для источников и приемников.  $f\left(\mathbf{r}_{*}\right)-$ искомое сейсмическое изображение в глубинной точке $\mathbf{r}_{*} ; u_{0}(\mathbf{s}, \mathbf{g}, t)-$ исходные данные многократных перекрытий для источников и приемников с координатами $(\mathbf{s}, \mathbf{g}) ; \tau_{s}+\tau_{g}-$ сумма времен пробега из $\mathbf{r}_{*}$ до источников и приемников;  $\nu$ и $k-$ заданные параметры среды, входящие в волновое уравнение (1) ; $H-$ преобразование Гильберта;  $\left[A_{s}\right]$ и $\left[A_{g}\right]$ величины лучевых амплитуд для источников и приемников. \\ 
Во внутреннем 4-кратном интеграле (8) по координатам источников и приемников можно так структурировать исходные данные, чтобы обеспечить миграционное преобразование в сейсмическое изображение однократных сейсмических кубов квазиравных удалений/азимутов направлений источники - приемники. С этой целью введем «биновые» координаты средних точек и половинных удалений: m $=(\mathbf{s}+\mathbf{g}) / 2$, $\mathbf{h}=(\mathbf{s}-\mathbf{g}) / 2 .$ Средние точки $\mathbf{m}=\mathbf{m}\left(m_{x}, m_{y}\right)$ образуют плотную и регулярную биновую сетку. Назовем «покрытием» совокупность исходных сейсмических трасс, однократно заполняющих узлы этой биновой сетки для некоторого фиксированного вектора $\mathbf{h}\left(h_{x}, h_{y}\right)$. С учетом сказанного перепишем аппроксимацию (8):\\
{\small 
$$f\left(\mathbf{r}_{*}\right) \approx\left(\frac{1}{4 \pi}\right)^{2} \sum_{l, l'} \int \limits_{S_{l}} \int \limits_{S_{l'}} k_{\mathrm{s}}k_{\mathrm{g}} \frac{\cos \left(\alpha_{s}\right)_{l} \cos \left(\alpha_{g}\right)_{l'}}{\nu_{s} \nu_{g}}\left[A_{s}\right]_{l}\left[A_{g}\right]_{l'} H^{\operatorname{Ind}_l+\operatorname{Ind}_{l'}}\left(u_{0}\right)_{t}^{\prime}\left(\mathbf{s}, \mathbf{g} ;\left(\tau_{s}\right)_{l}+\left(\tau_{g}\right)_{l'}\right) d \mathbf{s} d \mathbf{g} =$$ $$= \left(\frac{1}{2 \pi}\right)^{2} \sum \limits_{l, l'} \int\limits_{S_{l}} \int\limits_{S_l'} k_{s}k_g \frac{\cos \left(\alpha_{s}\right)_{l} \cos \left(\alpha_{g}\right)_{l'}}{\nu_{s} \nu_{g}}\left[A_{s}\right]_{l}\left[A_{g}\right]_{l'} H^{\operatorname{Ind}_{l}+\operatorname{Ind}_{l'}}\left(u_{0}\right)_{t}^{\prime}\left(\mathbf{m}+\mathbf{h}, \mathbf{m}-\mathbf{h} ;\left(\tau_{s}\right)_{l}+\left(\tau_{g}\right)_{l'}\right) d \mathbf{m} d \mathbf{h}=$$  \\ $=\left(\frac{1}{2 \pi}\right)^{2} \int d \mathbf{h} \left( \sum_{l, l'} \int \limits_{S_{l}} k_{\mathrm{s}} k_{\mathrm{g}} \frac{\cos \left(\alpha_{s}\right)_{l} \cos \left(\alpha_{g}\right)_{l'}}{\nu_{s} \nu_{g}}\left[A_{s}\right]_{l}\left[A_{g}\right]_{l'} H^{\operatorname{Ind}_{l}+\operatorname{Ind}_{l'}}\left(u_{0}\right)_{t}^{\prime}\left(\mathbf{m}+\mathbf{h}, \mathbf{m}-\mathbf{h} ;\left(\tau_{s}\right)_{l}+\left(\tau_{g}\right)_{l'}\right) d \mathbf{m}\right)$} (9)\\
Итоговое глубинное изображение $f(\vec r_*)$ представляется набором изображений $f(\vec r_*,\vec h)$ результатов миграции фиксированного покрытия h = const.
Выражение, стоящее в круглых скобках, представляет собой результат миграции фиксированного покрытия $\mathbf{h}=$ const, a внешний интеграл по h превращается в обычную сумму по покрытиям. Следуя известным представлениям о сейсмической миграции как об операторе, переводящем однократный сейсмический куб исходных данных в глубинный куб сейсмического изображения, была реализована возможность в качестве исходного куба данных подавать на вход многолучевой миграции однократные кубы квазиравных удалений/азимутов направлений источники – приемники. Число таких кубов, на которые заранее рассортируются исходные сейсмические данные, отвечает номинальной кратности съемки. После выполнения миграционного преобразования каждого исходного куба в глубинный куб сейсмического изображения выполняется сборка сейсмограмм общей точки изображения, далее – их осреднение с подбором мьютинга и получение суммарного куба сейсмического изображения. 
 \section{Суммирование}

Функция Грина от приемника $g$ или источника $s$  задана в узлах равномерной прямоугольной сетки $\{r^G_i\}$,
с шагами $\Delta_x^G, \Delta_y^G, \Delta_z^G$, ось $Z$ направлена {\bf вниз}.
Изображение вычисляется в узлах равномерной прямоугольной сетки $\{r^I_i\}$ с шагами $\Delta_x^I, \Delta_y^I, \Delta_z^I$,
совпадающей с сеткой задания фунцкции Грина по латерали,
и измельченной (по сравнению с сеткой задания фунцкции Грина) по вертикали в $n_Z$ раз, $n_Z\geq 1$,
таким образом размер сетки изображения по вертикали $N^I_z$
$$N^I_z = (N^G_z-1)n_Z+1, \qquad \Delta^I_{x,y} = \Delta^G_{x,y}, \qquad \Delta_z^I = \Delta_z^G/n_Z,$$
где $N_z^G$ --- размер сетки фунцкии Грина по вертикали.
Для вычисления изображения в произвольном узле c номером по глубине $z^I$  требуются функции Грина
от приемника $G^g$ и источника $G^s$ в соседних по вертикали узлах сверху $z^G_u = z^I/n_Z$
и снизу $z^G_d=z^I/n_Z+1$, второй узел нужен
только если $z^In_Z\neq z^G_u$.

В каждом узле $r^G$ может существовать
несколько треков (возможностей прохода сигнала из источника/приемника), характризующихся амплитудой $A$,
временем прохождения $\tau$, индексом~$I$ и парамерами окна фильтрации по времени
$$
{\mathbf\Delta}^{\tau} =(\Delta^{\tau x},  \Delta^{\tau y}) =
\left(\frac{\Delta_{{\rm bin}x} n_{0x}}{v_0}, \frac{\Delta_{{\rm bin}y} n_{0y}}{v_0} \right),
$$
где $\Delta_{{\rm bin}x,y}$ --- размеры площадки источника, $n_{0x,y}$ --- компоненты направления выхода луча из источника,
$v_0$ --- скорость в источнике в направлении луча, таким образом компоненты вектора~${\mathbf\Delta}^{\tau}$
  являются {\bf знакопеременными}.
  
  Тогда восстановление функции Грина на сетке изображения производится по формуле\\
$
G =
w_u \sum_i^{N_u} A_{u,i}\, {\widehat T}(|{\mathbf\Delta}^\tau_{u,i}|)\, {\widehat H}(I_{u,i})\, u_0\left(\tau_{u,i} +  \Delta_u {n_{zu}} \right) +
w_d \sum_i^{N_d} A_{d,i}\, {\widehat T}(|{\mathbf\Delta}^\tau_{d,i}|)\, {\widehat H}(I_{d,i})\, u_0\left(\tau_{d,i} -  \Delta_d {n_{zd}}\right)(10) $ \\
где $w_u= (z_d-z)/\Delta^G_z$, $w_d=1-w_u$ --- веса линейной интерполяции по вертикали,
$N_{u,d}$~--- число треков в верхнем и нижнем узлах, ${\widehat T}(\Delta^\tau)$~--- оператор фильтрации сигнала в окне
ширины $\Delta^\tau$, ${\widehat H}(I)$~--- оператор Гильберт--преобразования с индексом $I$,
{$n_{zu,d}$ --- $z$--компоненты направления распространения лучей в соотв. точках глубинной сетки (косинусы углов {\bf прихода} относительно вертикали)},
$\Delta_{u,d}$~--- задержки сигнала,
$$
\Delta_u = \frac{2 w_d \Delta_z^G}{v_u + v_c}, \qquad
\Delta_d = \frac{2 w_u \Delta_z^G}{v_d + v_c}, \qquad
v_c = w_d v_u + w_u v_d,
$$
$v_{u,d}$ --- скорости в верхнем и нижнем узлах.\\
При суммировании используется выражение\\
$
S =
w_u \sum_i^{N_u^s} \sum_j^{N_u^g} A_{u,i}^s A_{u,j}^g\,
{\widehat T}(|{\mathbf\Delta}^{\tau s}_{u,i} + {\mathbf\Delta}^{\tau g}_{u,j}|)\,
{\widehat H}(I_{u,i}^s + I_{u,j}^g)\, u_0\left(\tau_{u,i}^s + \tau_{u,j}^g +  (n_{zu,i}+n_{zu,j})\Delta_u \right) + \\ +
w_d \sum_i^{N_d^s} \sum_j^{N_d^g} A_{d,i}^s A_{d,j}^g\,
{\widehat T}(2|{\mathbf\Delta}^{\tau s}_{d,i} + {\mathbf\Delta}^{\tau g}_{d,j}|)\,
{\widehat H}(I_{d,i}^s + I_{d,j}^g)\, u_0\left(\tau_{d,i}^s + \tau_{d,j}^g -  (n_{zd,i}+n_{zd,j})\Delta_d \right).
 (11) $
 Рассмотрим построение Гильберт-преобразования и фильтрации в скользящем окне.
Пусть исходная сейсмическая трасса $u_0(t)$ представлена в виде кусочно--постояннной функции $\{u_{0i}\}$
с шагом дискретизации $\Delta_t$:
$$
u_0(t) = u_{0i}\quad\text{при}\quad  (i-1)\Delta_t\leq t < i\Delta_t.
$$
Ее Гильберт--преобразование $\{g_i\}$ для индекса единица можно записать как свертку c дискретным антисимметричным ядром $H_i$
полуширины $N_H=20$:
$$
g_i = \sum_{k=-N_H}^{N_H} u_{0i+k} H_k, \qquad H_{-k}=-H_k, \qquad H_0 = 0,
$$
{\small
\begin{tabular}{|c|c|c|c|c|}
  \hline
	$H_1=0.430652544$   &  $H_2=0.162448676$    & $H_3=0.098590612$   & $H_4=0.068790183$   & $H_5=0.051374347$   \\  
	$H_6=0.039912643$   &  $H_7=0.031785253$    & $H_8=0.025717292$   & $H_9=0.021011961$   & $H_{10}=0.017255605$ \\
	$H_{11}=0.014186911$ &  $H_{12}=0.011632588$ & $H_{13}=0.009473137$ & $H_{14}=0.007623455$ & $H_{15}=0.006021281$ \\ 
	$H_{16}=0.004620007$ &  $H_{17}=0.003384044$ & $H_{18}=0.002285748$ & $H_{19}=0.001303317$ & $H_{20}=0.000419327$ \\
    \hline
\end{tabular}
}

При расчете Гильберт--преобразования для произвольного индекса $I$ используется выражение:
$$
\widehat{H}(I) u_{0i} = \left\{\begin{matrix}
u_{0i} & \text{при} &  \mod(I,4)=0, \\
g_i & \text{при} &  \mod(I,4)=1, \\
-u_{0i} & \text{при} & \mod(I,4)=2, \\
-g_i & \text{при} & \mod(I,4)=3, \\
\end{matrix}\right.
$$
где $\mod$ --- остаток от деления.\\
Для ускорения оператора фильтрации в скользящем окне $\widehat T(\Delta^\tau)$ перейдем
от кусочно--постоянных $u_0$ и $g$ к их линейно--интерполированным первообразным $F$ и $G$:
\begin{align}
F(t) &= (1-q)F_{i-1}+q F_i, \quad q = t/\Delta_t-(i-1), \quad\text{при}\quad  (i-1)\Delta_t\leq t < i\Delta_t, \notag\\
G(t) &= (1-q)G_{i-1}+q G_i, \quad q = t/\Delta_t-(i-1), \quad\text{при}\quad  (i-1)\Delta_t\leq t < i\Delta_t, \notag\\
F_i &= \sum_{k=1}^i f_k, \qquad G_i = \sum_{k=1}^i g_k. \notag
\end{align}

Тогда
$$
\widehat{T}\left(\Delta^{\tau}\right) F(t)=\left\{\begin{array}{ccr}
\frac{F\left(t+\Delta^{\tau}\right)-F\left(t-\Delta^{\tau}\right)}{2 \Delta^{\tau}} & \text { при } & \text { floor }\left(\frac{t+\Delta^{\tau}}{\Delta_{t}}\right) \neq \text { floor }\left(\frac{t-\Delta^{\tau}}{\Delta_{t}}\right) \\
\frac{F_{i}-F_{i-1}}{\Delta_{t}} & \text { при } & i=\text { floor }\left(\frac{t+\Delta^{\tau}}{\Delta_{t}}\right)=\text { floor }\left(\frac{t-\Delta^{\tau}}{\Delta_{t}}\right)
\end{array}\right.
$$
\section{Постановка задачи}
Основной идеей миграции до суммирования на основе обращённого волнового продолжения является восстановление поля отражённых волн во всей среде $(z>0)$ по его граничным значениям при $z=0$.  Данный принцип состоит в том, что в качестве изображения среды в каждой точке r берётся значения продолженного поля отраженных волн на том времени, которое соответствует времени прихода в эту точку прямой волны от источника. Основной идеей обращённого волнового продолжения является переход от прямого времени к обращённому: $\tau = T-t$, где T определяет момент окончания регистрации. Записанные на поверхности данные рассматриваются как граничные условия. Также для решения уравнения необходимо задать начальные условия (в обращённом времени) ,описывающие все упругие волны в среде на момент окончания регистрации сейсмических колебаний. Поскольку такая информация недоступна, приходится пренебрегать их влиянием и полагать начальные условия равными нулю.  
\section{Программная реализация на Cuda}
\section{Результаты вычислительных экспериментов и их анализ}
\newpage 
\begin{thebibliography}{3}
\bibitem{}
Гогоненков Г.Н., Мороз Б.П., Плешкевич А.Л., Турчанинов В.И. Теоретические основы и практическое использование отечественной программы 3D-глубинной сейсмической миграции до суммирования. Геофизика. 2007. № 4. С. 15–24.
\bibitem{}
Плешкевич А.Л., Иванов А.В., Левченко В.Д., Хилков С. А. Многолучевая 3D-глубинная сейсмическая миграция до суммирования с сохранением амплитуд Геофизика.  2017.  № S (спец. выпуск). С. 76–84.
\bibitem{}
Плешкевич А.Л. Методы реконструкции изображения глубинных неоднородностей земной среды по сейсмическим данным («сейсмическая миграция»): дис. … доктор. ф.-м. наук: 1.6.9 – Геофизика / Плешкевич А.Л.  – НСК., 2021. – 216 с.
\bibitem{}
 Морс Ф. М., Фешбах Г.  Методы теоретической физики, т.1 – М.: ИЛ, 1958. – 931 с.
 \bibitem{}
 Боганик Г.Н., Гурвич И.И. Сейсмическая разведка – Тверь, АИС 2006 г., – 744 с.
 \bibitem{}
 Клаербоут Д. Ф. Сейсмическое изображение земных недр – М.: Недра 1989 г. –
407 с.
\bibitem{}
Воскресенский Ю.Н. Построение сейсмических изображений: Учебное пособие. - М.: РГУ нефти и газа им. И.М. Губкина, 2006. - 116 с.
\bibitem{}
Вайнберг, Б.Р. Асимптотические методы в уравнениях математической физики – М.: Изд-во Моск. ун-та, 1982. – 296 с.
\bibitem{}
Бронштейн И.Н., Семендяев  К.А.  Справочник по математике для инженеров и учащихся втузов – М.: Наука, 1986. – 544 с.
\bibitem{}
Козлов Е. А. Миграционные преобразования в сейсморазведке – M.: Недра, 1986. – 247 c.
\end{thebibliography}








\end{document}
